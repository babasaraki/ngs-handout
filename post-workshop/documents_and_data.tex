\section{Access to Workshop Documents}

This document has been written in \LaTeX\ and deposited in a public github
repository (\url{https://github.com/nathanhaigh/ngs_workshop}). The
documentation has been released under a Creative Commons Attribution 3.0
Unported License (see the Licence page at the beginning of this handout).

For convienience, you can access up-to-date PDF versions of the \LaTeX\ documents at:
\begin{description}[style=multiline,labelindent=0cm,align=left,leftmargin=0.5cm]
\item[Trainee Handout]\hfill\\
\url{https://github.com/downloads/nathanhaigh/ngs_workshop/trainee_handout_latest.pdf}
\item[Trainer Handout]\hfill\\
\url{https://github.com/downloads/nathanhaigh/ngs_workshop/trainer_handout_latest.pdf}
\end{description}

\section{Access to Workshop Data}
Once you have created a VM from our image file, either locally using VirtualBox
or on the NeCTAR Research Cloud, you can configure the system with the workshop
documents and data. This way you can revisit and work through this workshop in
your own time.

In order to do this, we have provided you with access to a shell script which
should be executed on your NGS Training VM by the \texttt{ubuntu} user. This pulls
approx. 3.3 GBytes of data from the NeCTAR Cloud storage and configures the system
for running this workshop:

% NOTE This bash script could be entered into the user data when instantiating
% the VM in the first place
\begin{lstlisting}
# As the ubuntu user run the following commands:
cd /tmp
wget https://github.com/nathanhaigh/ngs_workshop/raw/master/\
workshop_setup/setup_NGS_workshop.sh
bash setup_NGS_workshop.sh
\end{lstlisting}

While you're at it, you may also like to change the timezone of your VM to match
that of your own. To do this simply run the following commands as the
\texttt{ubuntu} user:
\begin{lstlisting}
TZ="Australia/Adelaide"
echo "$TZ" | sudo tee /etc/timezone
sudo dpkg-reconfigure --frontend noninteractive tzdata
\end{lstlisting}

For further information about what this script does and possible command line
arguments, see the script's help:
\begin{lstlisting}
bash setup_NGS_workshop.sh -h
\end{lstlisting}


For further information about setting up the VM for the workshop, please see:
\\\\
\url{https://github.com/nathanhaigh/ngs_workshop/blob/master/workshop_setup/README.md}
