% Define the top matter
\renewcommand{\moduleTitle}{Read Alignment}
\renewcommand{\moduleAuthors}{%
  Myrto Kostadima \mailto{kostadim@ebi.ac.uk}
} \renewcommand{\moduleContributions}{%
  Xi Li \mailto{sean.li@csiro.au}%
}

%  Start: Module Title Page
\chapter{\moduleTitle}
\newpage
% End: Module Title Page

\section{Key Learning Outcomes}

After completing this practical the trainee should be able to:
\begin{itemize}
  \item Perform the simple NGS data alignment task against one interested reference data
  \item Interpret and manipulate the mapping output using SAMtools
  \item Visualise the alignment via a standard genome browser, e.g. IGV browser
\end{itemize}

\section{Resources You'll be Using}
 
\subsection{Tools Used}
\begin{description}[style=multiline,labelindent=0cm,align=left,leftmargin=0.5cm]
  \item[Bowtie]\hfill\\
  	\url{http://bowtie-bio.sourceforge.net/index.shtml}
  \item[Bowtie 2]\hfill\\
  	\url{http://bowtie-bio.sourceforge.net/bowtie2/index.shtml}
  \item[Samtools]\hfill\\
  	\url{http://picard.sourceforge.net/}
  \item[BEDTools]\hfill\\
  	\url{http://code.google.com/p/bedtools/}
  \item[UCSC tools]\hfill\\
  	\url{http://hgdownload.cse.ucsc.edu/admin/exe/}  
  \item[IGV genome browser]\hfill\\
  	\url{http://www.broadinstitute.org/igv/}
\end{description}

\section{Useful Links}
 
\begin{description}[style=multiline,labelindent=0cm,align=left,leftmargin=0.5cm]
  \item[SAM Specification]\hfill\\
    \url{http://samtools.sourceforge.net/SAM1.pdf}
  \item[Explain SAM Flags]\hfill\\
    \url{http://picard.sourceforge.net/explain-flags.html}
\end{description}

\subsection{Sources of Data}
% TODO more specific about how the data was derived for this module
  \url{http://www.ebi.ac.uk/arrayexpress/experiments/E-GEOD-11431}
% \url{http://www.ebi.ac.uk/ena/data/view/ERR022484}\\
% \url{http://www.ebi.ac.uk/ena/data/view/ERR022485}

\clearpage

\section{Introduction}

\begin{information}
The goal of this hands-on session is to perform an unspliced alignment for a
small subset of raw reads. We will align raw sequencing data to the mouse genome
using Bowtie and then we will manipulate the SAM output in order to
visualize the alignment on the IGV browser.
\end{information}

\section{Prepare the Environment}

\begin{information}
We will use one data set in this practical, which can be found in the \texttt{ChIP-seq}
directory on your desktop.
\end{information}

\begin{steps}
Open the Terminal.

First, go to the right folder, where the data are stored.
\begin{lstlisting}
cd ~/ChIP-seq
\end{lstlisting}

\begin{information}
The \texttt{.fastq} file that we will align is called \texttt{Oct4.fastq}. This
file is based on Oct4 ChIP-seq data published by Chen \textit{et al.} (2008). For the
sake of time, we will align these reads to a single mouse chromosome.
\end{information}
\end{steps}

\section{Alignment}

\begin{information}
You already know that there are a number of competing tools for short read
alignment, each with its own set of strengths, weaknesses, and caveats. Here we
will try Bowtie, a widely used ultrafast, memory efficient short read
aligner.
\end{information}

\begin{steps}
Bowtie has a number of parameters in order to perform the alignment. To
view them all type

\begin{lstlisting}
bowtie --help
\end{lstlisting}

Bowtie uses indexed genome for the alignment in order to keep its memory
footprint small. Because of time constraints we will build the index only for
one chromosome of the mouse genome. For this we need the chromosome sequence in
FASTA format. This is stored in a file named \texttt{mm9}, under the subdirectory
\texttt{bowtie\_index}.

The indexed chromosome is generated using the command:

\begin{lstlisting}
bowtie-build bowtie_index/mm9.fa bowtie_index/mm9
\end{lstlisting}

This command will output 6 files that constitute the index. These files that
have the prefix \texttt{mm9} are stored in the \texttt{bowtie\_index}
subdirectory. To view if they files have been successfully created type:

\begin{lstlisting}
ls -l bowtie_index
\end{lstlisting}
\end{steps}

\begin{information}
Now that the genome is indexed we can move on to the actual alignment. The first
argument for \texttt{bowtie} is the basename of the index for the genome to be searched;
in our case this is \texttt{mm9}. We also want to make sure that the output is in SAM
format using the \texttt{-S} parameter. The last argument is the name of the
FASTQ file.
\end{information}

\begin{steps}
Align the Oct4 reads using Bowtie: 

\begin{lstlisting}
bowtie bowtie_index/mm9 -S Oct4.fastq > Oct4.sam
\end{lstlisting}

The above command outputs the alignment in SAM format and stores them in the
file \texttt{Oct4.sam}.
\end{steps}

\begin{note}
In general before you run Bowtie, you have to know what quality encoding your FASTQ files
are in. The available FASTQ encodings for bowtie are:

\begin{description}[style=multiline,labelindent=0cm,align=right,leftmargin=\descriptionlabelspace,rightmargin=1.5cm,font=\ttfamily]
 \item[--phred33-quals] Input qualities are Phred+33 (default).
 \item[--phred64-quals] Input qualities are Phred+64 (same as \texttt{--solexa1.3-quals}).
 \item[--solexa-quals] Input qualities are from GA Pipeline ver. $<$ 1.3.
 \item[--solexa1.3-quals] Input qualities are from GA Pipeline ver. $\geq$ 1.3.
 \item[--integer-quals] Qualities are given as space-separated integers (not ASCII).
\end{description}

The FASTQ files we are working with are Sanger encoded (Phred+33), which is the
default for Bowtie.

Bowtie will take 2-3 minutes to align the file. This is fast compared to
other aligners which sacrifice some speed to obtain higher sensitivity.
\end{note}

\begin{steps}
Look at the top 10 lines of the SAM file by typing:

\begin{lstlisting}
head -n 10 Oct4.sam
\end{lstlisting}
\end{steps}

\begin{questions}
Can you distinguish between the header of the SAM format and the actual alignments?
\begin{answer}
The header line starts with the letter `@', i.e.: 

\begin{tabular}{llll}
@HD & VN:1.0 & SO:unsorted & \\
@SQ & SN:chr1 & LN:197195432 & \\
@PG & ID:Bowtie &      VN:0.12.8  & CL:``bowtie bowtie\_index/mm9 -S Oct4.fastq'' \\
\end{tabular}

While, the actual alignments start with read id, i.e.:

\begin{tabular}{llll}
SRR002012.45 & 0 & chr1 & etc \\
SRR002012.48 & 16 & chr1 & etc \\
\end{tabular}
\end{answer}

What kind of information does the header provide?
\begin{answer}
\begin{itemize}
  \item @HD: Header line; VN: Format version; SO: the sort order of alignments.
  \item @SQ: Reference sequence information; SN: reference sequence name; LN: reference sequence length.
  \item @PG: Read group information; ID: Read group identifier; VN: Program version; CL: the command line that produces the alignment.
\end{itemize}
\end{answer}

To which chromosome are the reads mapped? 
\begin{answer}
Chromosome 1.
\end{answer}
\end{questions}

\section{Manipulate SAM output}

\begin{note}
SAM files are rather big and when dealing with a high volume of NGS data,
storage space can become an issue. As we have already seen, we can convert SAM
to BAM files (their binary equivalent that are not human readable) that occupy
much less space.
\end{note}

\begin{steps}
Convert SAM to BAM using \texttt{samtools view} and store the output in the file
\texttt{Oct4.bam}. You have to instruct \texttt{samtools view} that the input is in SAM
format (\texttt{-S}), the output should be in BAM format (\texttt{-b}) and that
you want the output to be stored in the file specified by the \texttt{-o}
option:

\begin{lstlisting}
samtools view -bSo Oct4.bam Oct4.sam
\end{lstlisting}
\end{steps}

\begin{advanced}
Compute summary stats for the Flag values associated with the alignments using:

\begin{lstlisting}
samtools flagstat Oct4.bam
\end{lstlisting}
\end{advanced}

\section{Visualize alignments in IGV}

\begin{information}
IGV is a stand-alone genome browser. Please check their website
(\url{http://www.broadinstitute.org/igv/}) for all the formats that IGV
can display. For our visualization purposes we will use the BAM and bigWig
formats.
\end{information}

\begin{note}
When uploading a BAM file into the genome browser, the browser will look for the
index of the BAM file in the same folder where the BAM files is. The index file
should have the same name as the BAM file and the suffix \texttt{.bai}. Finally, to
create the index of a BAM file you need to make sure that the file is sorted
according to chromosomal coordinates.
\end{note}

\begin{steps}
Sort alignments according to chromosomal position and store the result in the
file with the prefix \texttt{Oct4.sorted}:

\begin{lstlisting}
samtools sort Oct4.bam Oct4.sorted
\end{lstlisting}

Index the sorted file.

\begin{lstlisting}
samtools index Oct4.sorted.bam
\end{lstlisting}

The indexing will create a file called \texttt{Oct4.sorted.bam.bai}. Note that
you don't have to specify the name of the index file when running
\texttt{samtools index}, it simply appends a \texttt{.bai} suffix to the input
BAM file.
\end{steps}

\begin{note}
Another way to visualize the alignments is to convert the BAM file into a bigWig
file. The bigWig format is for display of dense, continuous data and the data
will be displayed as a graph. The resulting bigWig files are in an indexed
binary format.
\end{note}

\begin{steps}
The BAM to bigWig conversion takes place in two steps. Firstly, we convert the
BAM file into a bedgraph, called \texttt{Oct4.bedgraph}, using the tool
\texttt{genomeCoverageBed} from BEDTools. Then we convert the bedgraph into a
bigWig binary file called \texttt{Oct4.bw}, using \texttt{bedGraphToBigWig} from
the UCSC tools:

\begin{lstlisting}
genomeCoverageBed -bg -ibam Oct4.sorted.bam -g bowtie_index/mouse.mm9.genome > Oct4.bedgraph
bedGraphToBigWig Oct4.bedgraph bowtie_index/mouse.mm9.genome Oct4.bw
\end{lstlisting}
\end{steps}

\begin{note}
Both of the commands above take as input a file called \texttt{mouse.mm9.genome} that
is stored under the subdirectory \texttt{bowtie\_index}. These genome files are
tab-delimited and describe the size of the chromosomes for the organism of
interest. When using the UCSC Genome Browser, Ensembl, or Galaxy, you typically
indicate which species/genome build you are working with. The way you do this for
BEDTools is to create a ``genome'' file, which simply lists the names of
the chromosomes (or scaffolds, etc.) and their size (in basepairs).

BEDTools includes pre-defined genome files for human and mouse in the
\texttt{genomes} subdirectory included in the BEDTools distribution.
\end{note}

\begin{steps}
Now we will load the data into the IGV browser for visualization. In order to
launch IGV double click on the \texttt{IGV 2.1} icon on your Desktop. Ignore any warnings
and when it opens you have to load the genome of interest.

On the top left of your screen choose from the drop down menu \texttt{Mus musculus
(mm9)}. Then in order to load the desire files go to:

\begin{lstlisting}
File > Load from File
\end{lstlisting}

On the pop up window navigate to Desktop $>$ ChIP-seq folder and select the file
\texttt{Oct4.sorted.bam}.

Repeat these steps in order to load \texttt{Oct4.bw} as well.

Select \texttt{chr1} from the drop down menu on the top left. Right click on the name of
\texttt{Oct4.bw} and choose Maximum under the Windowing Function. Right click again and
select Autoscale.

In order to see the aligned reads of the BAM file, you need to zoom in to a
specific region. For example, look for gene \texttt{Lemd1} in the search box.
\end{steps}

\begin{questions}
What is the main difference between the visualization of BAM and bigWig files?
\begin{answer}
The actual alignment of reads that stack to a particular region can be displayed
using the information stored in a BAM format.
The bigWig format is for display of dense, continuous data that will be
displayed in the Genome Browser as a graph.
\end{answer}
\end{questions}

Using the \texttt{+} button on the top right, zoom in to see more of the details of the alignments.

\begin{questions}
What do you think the different colors mean?
\begin{answer}
The different color represents four nucleotides, e.g. blue is Cytidine (C), red
is Thymidine (T).
\end{answer}
\end{questions}

\section{Practice Makes Perfect!}
\begin{steps}
In the ChIP-seq folder you will find the file \texttt{gfp.fastq}. Follow the
above described analysis, from the bowtie alignment step, for this dataset as
well. You will need these files for the ChIP-Seq module.
\end{steps}
