% Define the top matter
\renewcommand{\moduleTitle}{ChIP-Seq}
\renewcommand{\moduleAuthors}{%
  Remco Loos, EMBL-EBI \mailto{remco@ebi.ac.uk} \\
  Myrto Kostadima \mailto{kostadim@ebi.ac.uk}
} \renewcommand{\moduleContributions}{%
  Xi Li \mailto{sean.li@csiro.au}%
}

%  Start: Module Title Page
\chapter{\moduleTitle}
\newpage
% End: Module Title Page

\section{Key Learning Outcomes}

After completing this practical the trainee should be able to:
\begin{itemize}
  \item Perform simple ChIP-Seq analysis, e.g. the detection of immuno-enriched areas using the chosen peak caller program MACS
  \item Visualize the peak regions through a genome browser, e.g. Ensembl, and identify the real peak regions
  \item Perform functional annotation and detect potential binding sites (motif) in the predicted binding regions using motif discovery tool, e.g. MEME.
\end{itemize}

\section{Resources You'll be Using}
 
\subsection{Tools Used}
\begin{description}[style=multiline,labelindent=0cm,align=left,leftmargin=0.5cm]
  \item[MACS]\hfill\\
  	\url{http://liulab.dfci.harvard.edu/MACS/index.html}
  \item[Ensembl]\hfill\\
  	\url{http://www.ensembl.org}
  \item[PeakAnalyzer]\hfill\\
  	\url{http://www.ebi.ac.uk/bertone/software}
  \item[MEME]\hfill\\
  	\url{http://meme.sdsc.edu/meme/cgi-bin/meme.cgi}
  \item[TOMTOM]\hfill\\
  	\url{http://meme.sdsc.edu/meme/cgi-bin/tomtom.cgi}  
  \item[DAVID]\hfill\\
  	\url{http://david.abcc.ncifcrf.gov}
  \item[GOstat]\hfill\\
    \url{http://gostat.wehi.edu.au}
\end{description}

\subsection{Sources of Data}
  \url{http://www.ebi.ac.uk/arrayexpress/experiments/E-GEOD-11431}
 % These data are reported in Chen, X et al. (2008) Integration of external signaling pathways with the core tran-scriptional network in embryonic stem cells. Cell. Jun 13;133(6):1106-17.
% \url{http://www.ebi.ac.uk/ena/data/view/ERR022484}\\
% \url{http://www.ebi.ac.uk/ena/data/view/ERR022485}

\newpage

\section{Introduction}

\begin{information}
The goal of this hands-on session is to perform some basic tasks in the analysis
of ChIP-seq data. In fact, you already performed the first step, alignment of
the reads to the genome, in the previous session. We start from the aligned
reads and we will find immuno-enriched areas using the peak caller MACS. We will
visualize the identified regions in a genome browser and perform functional
annotation and motif analysis on the predicted binding regions.
\end{information}

\section{Prepare the Environment}

\begin{information}
The material for this practical can be found in the ChIP-seq directory on your
desktop. This directory also contains an electronic version of this document,
which can be useful to copy and paste commands. Please make sure that this
directory also contains the SAM/BAM files you produced during the alignment
practical.
\end{information}

\begin{steps}
If you didn't have time to align the control file called \texttt{gfp.fastq} during the
alignment practical, please do it now.

Follow the same steps as for the \texttt{Oct4.fastq} file.
\end{steps}

\begin{information}
Of course, the same bowtie index can be used, so the index building step can be
skipped.
\end{information}

\begin{note}
In ChIP-seq analysis (unlike in other applications such as RNA-seq) we often
would like to exclude all reads that map to more than one location in the
genome, to avoid false positives. When using Bowtie, this can be done using the
\texttt{-m 1} option, which tells it to report only unique matches (See
\texttt{bowtie --help} for more details).
\end{note}


\begin{steps}
Open the Terminal and go to the \texttt{ChIP-seq} directory:
\begin{lstlisting}
cd ~/ChIP-seq
\end{lstlisting}
\end{steps}

\section{Finding enriched areas using MACS}

\begin{information}
MACS stands for Model based analysis of ChIP-seq. It was designed for
identifying transcription factor binding sites. MACS captures the influence of
genome complexity to evaluate the significance of enriched ChIP regions, and
improves the spatial resolution of binding sites through combining the
information of both sequencing tag position and orientation. MACS can be easily
used for ChIP-Seq data alone, or with a control sample to increase specificity.
\end{information}

\begin{steps}
Consult the MACS help file to see the options and parameters:

\begin{lstlisting}
macs --help
\end{lstlisting}
\end{steps}

\begin{information}
The input for MACS can be in ELAND, BED, SAM, BAM or BOWTIE formats (you just
have to set the \texttt{--format} option).

Options that you will have to use include: 

\begin{description}[style=multiline,labelindent=0cm,align=right,leftmargin=\descriptionlabelspace,rightmargin=1.5cm,font=\ttfamily]
 \item[-t] To indicate the input ChIP file.
 \item[-c] To indicate the name of the control file.
 \item[--format] To change the file format. The default format is bed..
 \item[--name] To set the name of the output files.
 \item[--gsize] This is the mappable genome size. With the read length we have,
 $70\%$ of the genome is a fair estimation. Since in this analysis we include
 only reads from chromosome 1, we will use as gsize $70\%$ of the length of
 chromosome 1 (197 Mb).
 \item[--tsize] To set the read length (look at the fastq files to check the
 length).
 \item[--wig] To generate signal wig files for viewing in a genome browser.
 Since this process is time consuming, it is recommended to run MACS first with
 this flag off, and once you decide on the values of the parameters, run MACS
 again with this flag on.
 \item[--diag] To generate a saturation table, which gives an indication whether
 the sequenced reads give a reliable representation of the possible peaks.
\end{description}
\end{information}

\begin{steps}
Now run macs using the following command:

\begin{lstlisting}
macs -t [Oct4 aligned bam file] -c [gfp aligned bam file] --format=BAM --name=Oct4 --gsize=138000000 --tsize=26 --diag --wig 
\end{lstlisting}

Look at the output saturation table (\texttt{Oct4 diag.xls}). To open Excel
files, right-click on them, choose Open with and select OpenOffice. Do you think
that more sequencing is necessary?

Open the Excel peak file and view the peak details. Note that the number of tags
(column 6) refers to the number of reads in the whole peak region and not the
summit height.

\end{steps}

\section{Viewing results with the Ensembl genome browser}

\begin{information}
It is often instructive to look at your data in a genome browser. Before, we
used IGV, a stand-alone browser, which has the advantage of being installed
locally and providing fast access. Web-based genome browsers, like Ensembl or
the UCSC browser, are slower, but provide more functionality. They do not only
allow for more polished and flexible visualisation, but also provide easy access
to a wealth of annotations and external data sources. This makes it
straightforward to relate your data with information about repeat regions, known
genes, epigenetic features or areas of cross-species conservation, to name just
a few. As such, they are useful tools for exploratory analysis.

They will allow you to get a `feel' for the data, as well as detecting
abnormalities and problems. Also, exploring the data in such a way may give you
ideas for further analyses.
\end{information}

\begin{steps}
Launch a web browser and go to the Ensembl website at
\url{http://www.ensembl.org/index.html}

Choose the genome of interest (that is, mouse) on the left side of the page.

Click on the \textbf{Manage your data} link on the left, then choose
\textbf{Attach remote file}.

\end{steps}

\begin{note}
One option is to browse to the wig file MACS generated (usually under MACS
wiggle/treat) and to upload it to Ensembl (option \textbf{Upload data}).
However, since wig files describing sequencing signal are very big, they would
take a long time to upload, and the browsing process will be very slow.

As a better alternative, wig files can be converted to an indexed binary format
and put into a web accessible server (HTTP, HTTPS, or FTP) instead on the
Ensembl server. This makes all the browsing process much faster. Detailed
instructions for generating a bigWig from a wig type file can be found at:

\url{http://genome.ucsc.edu/goldenPath/help/bigWig.html}.

\end{note}

\begin{steps}
We have generated bigWig files in advance for you to upload to the Ensembl
browser. They are at the following URL:
\url{http://www.ebi.ac.uk/~remco/ChIP-Seq_course/Oct4.bw}. To visualise the
data:
\begin{itemize}
	\item Paste the location above in the field File URL. 
	\item Choose data format Bigwig. 
	\item Choose some informative name and in the next window choose the colour of your preference. 
	\item Click \textbf{Save} and close the window to return to the genome browser. 
\end{itemize}
Repeat the process for the control sample, located at
\url{http://www.ebi.ac.uk/~remco/ChIP-Seq_course/gfp.bw}.
 
After uploading, choose \textbf{Configure this page}, and under \textbf{Your
data} tick both boxes. Closing the window will save these changes.

Go to a region on chromosome 1 (e.g. \texttt{1:34823162-35323161}), and zoom in and out
to view the signal and peak regions. Be aware that the data scale automatically,
so bigger-looking peaks need not actually be bigger. Always look at the values
on the left hand side axis.
\end{steps}

\begin{questions}
What can you say about the profile of Oct4 peaks? 
\begin{answer}
There is no significant peaks of Oct4 over the selected region.
\end{answer}

Compare it with H3K4me3 histone modification wig file we have generated at 
\url{http://www.ebi.ac.uk/~remco/ChIP-Seq_course/H3K4me3.bw}. 
\begin{answer}
H3K4me3 has a region that contains relatively high peaks than Oct4. 
\end{answer}

Jump to \texttt{1:36066594-36079728} for a sample peak. Do you think H3K4me3
peaks regions contain one or more modification sites? What about Oct4?
\begin{answer}
Yes. There are roughly three peaks, which indicate the possibility of having more than one modification sites in this region. 

For Oct4, no peak can be observed.
\end{answer}

\end{questions}

\begin{note}
MACS generates its peak files in a file format called bed file. This is a simple
text format containing genomic locations, specified by chromosome, begin and end
positions, and some more optional information.

See \url{http://genome.ucsc.edu/FAQ/FAQformat.html#format1} for details.

Bed files can also be uploaded to the Ensembl browser.
\end{note}

\begin{advanced}
Try uploading the peak file generated by MACS to Ensembl. Find the first peak in
the file (use the \texttt{head} command to view the beginning of the bed file), and see
if the peak looks convincing to you.
\end{advanced}


\section{Annotation: From peaks to biological interpretation}

\begin{information}
In order to biologically interpret the results of ChIP-seq experiments, it is
usually recommended to look at the genes and other annotated elements that are
located in proximity to the identified enriched regions. This can be easily done
using PeakAnalyzer.
\end{information}

\begin{steps}
Go to the PeakAnalyzer directory and launch the program by typing:
\begin{lstlisting}
java -jar PeakAnalyzer.jar &
\end{lstlisting}

The first window allows you to choose between the split application (which we
will try next) and peak annotation. Choose the peak annotation option and click
\textbf{Next}.

We would like to find the closest downstream genes to each peak, and the genes
that overlap with the peak region. For that purpose you should choose the
\textbf{NDG} option and click \textbf{Next}.

Fill in the location of the peak file \texttt{Oct4 peaks.bed}, and choose the mouse GTF
as the annotation file. You don't have to define a symbol file since gene
symbols are included in the GTF file.

Choose the output directory and run the program.
\end{steps}

\begin{information}
When the program has finished running, you will have the option of generating
plots, by pressing \textbf{Generate plots} (You can only do this if R is
installed on your computer, as is the case here. Otherwise, if you don't want to
install R, you can generate similar plots with Excel using the output files that
were generated by PeakAnalyzer.). A PDF file with the plots will be generated in
the output folder.
\end{information}

\begin{note}
This list of closest downstream genes (contained in the file
\texttt{Oct4\_peaks.ndg.bed}) can be the basis of further analysis. For instance,
you could look at the Gene Ontology terms associated with these genes to get an
idea of the biological processes that may be affected. Web-based tools like
DAVID (\url{http://david.abcc.ncifcrf.gov}) or GOstat
(\url{http://gostat.wehi.edu.au}) take a list of genes and return the enriched
GO categories.
\end{note}


\section{Motif analysis}

\begin{information}
It is often interesting to find out whether we can associate identified the
binding sites with a sequence pattern or motif. We will use MEME for motif
analysis. The input for MEME should be a file in FASTA format containing the
sequences of interest. In our case, these are the sequences of the identified
peaks that probably contain Oct4 binding sites.

Since many peak-finding tools merge overlapping areas of enrichment, the
resulting peaks tend to be much wider than the actual binding sites.
Sub-dividing the enriched areas by accurately partitioning enriched loci into a
finer-resolution set of individual binding sites, and fetching sequences from
the summit region where binding motifs are most likely to appear enhances the
quality of the motif analysis. Sub-peak summit sequences can be retrieved
directly from the Ensembl database using PeakAnalyzer.
\end{information}

\begin{steps}
If you have closed the PeakAnalyzer running window, open it again. If it is
still open, just go back to the first window.

Choose the split peaks utility and click \textbf{Next}. The input consists of
files generated by most peak-finding tools: a file containing the chromosome,
start and end locations of the enriched regions, and a \texttt{.wig} signal file
describing the size and shape of each peak. Fill in the location of both files
\texttt{Oct4 peaks.bed} and the wig file generated by MACS, which is under
\texttt{Oct4\_MACS\_wiggle/treat}, check the option to \textbf{Fetch subpeak
sequences} and click \textbf{Next}.

In the next window you have to set some parameters for splitting the peaks.

Separation float - This value determines when a peak will be separated into
sub-peaks. This is the ratio between a valley and its neighbouring summit (the
lower summit of the two). For example, if you set this height to be 0.5, two
sub-peaks will be separated only if the height of the lower summit is twice the
height of the valley. Keep the default value.

Minimum height - only sub-peaks with at least this number of tags in their
summit region will be separated. Set this to be 5. Change the organism name from
the default human to mouse and run the program.
\end{steps}

\begin{information}
Since the program has to read large wig files, it will take a few minutes to
run. Once the run is finished, two output files will be produced. The first
describes the location of the sub-peaks, and the second is a FASTA file
containing 300 sequences of length 61 bases, taken from the summit regions of
the highest sub-peaks.
\end{information}

\begin{steps}
Open a web bowser and go to the MEME website at
\url{http://meme.ebi.edu.au/meme/cgi-bin/meme.cgi}, and fill in the necessary
details, such as:
\begin{itemize}
	\item your e-mail address 
	\item the sub-peaks FASTA file \texttt{Oct4 peaks.bestSubPeaks.fa} (will need uploading), or just paste in the sequences. 
	\item the number of motifs we expect to find (1 per sequence) 
	\item the width of the desired motif (between 6 to 20) 
	\item the maximum number of motifs to find (3 by default). For Oct4 one classical motif is known. 
\end{itemize}
\end{steps}

\begin{note}
You will receive the results by e-mail. This usually doesn't take more than a few minutes.
\end{note}

\begin{steps}
Open the e-mail and click on the link that leads to the HTML results page.

Scroll down until you see the first motif logo. We would like to know if this
motif is similar to any other known motif. We will use TOMTOM for this. Scroll
down until you see the option \textbf{Submit this motif to}. Click the TOMTOM
button to compare to known motifs in motif databases, and on the new page choose
to compare your motif to those in the JASPAR and UniPROBE database.
\end{steps}


\begin{questions}
Which motif was found to be the most similar to your motif?
\begin{answer}
Sox2
\end{answer}
\end{questions}

\newpage
\section{Reference}
%TODO Use BibTeX
Chen, X et al.: Integration of external signaling pathways with the core transcriptional network in embryonic stem cells. Cell 133:6, 1106-17 (2008).
